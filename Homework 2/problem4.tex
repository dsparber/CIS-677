\section*{Problem 4}

\subsection*{Simple randomized algorithm}

\underline{Algorithm}

\begin{enumerate}
    \item Assign every $v \in V$ with $p = \frac{1}{2}$ to $U$
    \item Return $U$
\end{enumerate}

\underline{Analysis}

By definition, there is some maximum directed cut in $G$ (not necessarily unique).
Let $E_{max}$ be the set of all edges in that cut.

The probability of any edge $(i, j) \in E_{max}$ being in the cut that the algorithm generates, can be calculated as following:

$\Pr[\textit{(i, j) part of cut}] = 
\Pr[i \in U] \cdot \Pr[j \in W] = 
\Pr[i \in U] \cdot \Pr[j \notin U]= 
\frac{1}{2} \cdot \frac{1}{2} = 
\frac{1}{4}$

Thus, this algorithm yields a 4-approximation.

\subsection*{Integer linear program}

$z_{ij}$ is $1$ if the edge $(i, j) \in E$ is part of the cut, otherwise $0$.

$x_i$ is $1$ if $i \in U$, otherwise $0$.

Maximizing $\sum\limits_{(i,j)\in E} z_{ij}$ will therefore maximize the number of edges going from $U$ to $W$.

Every $i \in V$ is either part of $U$ or not, there is no in between. This is enforced by $\forall i \in V, x_i \in \{0, 1\}$.

$z_{ij}$ can be $1$ only if $i \in U$ and $j \in W$, otherwise it needs to be 0. This is strictly enforced by the three constraints $\forall (i, j) \in E, z_{ij} \leq x_i$, $\forall (i, j) \in E, z_{ij} \leq 1 - x_j$ and $\forall (i, j) \in E, 0 \leq z_{ij} \leq x_i$ combined with the fact, that $x_i \in \{0, 1\}$

Therefore the ILP correctly models the maximum directed cut problem


\subsection*{LP relaxation}

The constraint $\forall i \in V, x_i \in \{0, 1\}$ is changed to $\forall i \in V, x_i \in [0, 1]$.

\textit{Claim.} 
The integrality gap for the complete graph graph converges to 2, as $|V| \rightarrow \infty$.

\textit{Proof.}
For the complete graph, assigning $x_i = 0.5, i \in V$ and assigning $z_{i,j} = 0.5, (i, j) \in E$ for the LP relaxed problem yields $\frac{|E|}{2}$. 

However, the best ILP solution is assigning half of the nodes to $U$, which converges to $\frac{|E|}{4}$ for the complete graph, as $|V| \rightarrow \infty$.

Thus the integrality gap equals 2.

\subsection*{Rounding scheme}

The proposed rounding scheme returns a valid directed cut by construction, since every vertex is either assigned to $U$ or not $U$ (i.e. $W$). 

Let $(i, j) \in E$ be arbitrary.

$
\Pr[(i, j) \textit{ in cut}] = 
\Pr[i \in U, j \notin U]
$

$ =
\left(\frac{1}{4} + \frac{x_i}{2}\right) \cdot \left[1 - \left(\frac{1}{4} + \frac{x_j}{2}\right)\right] =
\left(\frac{1}{4} + \frac{x_i}{2}\right) \cdot \left(\frac{3}{4} - \frac{x_j}{2}\right) =
\left(\frac{1}{4} + \frac{x_i}{2}\right) \cdot \left(\frac{1}{4} + \frac{1 - x_j}{2}\right) 
$

$ \geq
\left(\frac{1}{4} + \frac{z_{ij}}{2}\right) \cdot \left(\frac{1}{4} + \frac{z_{ij}}{2}\right) = 
\frac{1}{16} + \frac{z_{ij}}{4} + \frac{z_{ij}^2}{4} = 
\frac{1}{16} - \frac{z_{ij}}{4} + \frac{z_{ij}^2}{4} + \frac{z_{ij}}{2} =
\left(\frac{1}{4} - \frac{z_{ij}}{2}\right)^2  + \frac{z_{ij}}{2} 
$

$\geq
\frac{z_{ij}}{2} 
$

With this probability the expected value of number of edges going from U to W can be calculated.

$
\EV[\textit{Total number of edges going from U to W}] = 
\sum\limits_{(i, j) \in E}  1\cdot Pr[(i, j) \textit{ in cut}]
$

$ \geq
\sum\limits_{(i, j) \in E} \frac{z_{ij}}{2} = 
\frac{1}{2} \cdot \underbrace{\sum\limits_{(i, j) \in E} z_{ij}}_{OPT_{LP}} \geq
\frac{1}{2} \cdot OPT
$

Thus, the proposed scheme yields a $2$-approximation. Therefore, the integrality gap is at most 2.

\pagebreak 