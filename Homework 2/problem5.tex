\section*{Problem 5}

We use the same LP relaxation as described in the lecture. 

Let $j \in [1..n]$ be an arbitrary element.

From the lecture, we know:

$\Pr[j \textit{ is not covered}] \leq \frac{1}{e}$

Instead of repeating $\log n$ times to get a valid solution, we now use a different approach:

Repeat $\log \Delta$ times to get a better probability: 

$
\Pr[j \textit{ is not covered after } \Delta \textit{ tries}] \leq
\left(\frac{1}{e}\right)^{\log \Delta} = 
\frac{1}{\Delta}
$


After that, for every uncovered element, pick the set with the smallest weight (greedy). 

Suppose $OPT = \{S_1,\dots,S_k\}$ is an optimal solution. Every $S_i$ of the solution covers at most $\Delta$ elements. Thus, the cost of picking sets by the greedy step is at most $\Delta$ times worse than the optimal solution.

Therefore, in total the expected cost increases by $\Delta \cdot \frac{1}{\Delta} = 1$ when doing the greedy step. We get a $(\log \Delta)$-approximation.