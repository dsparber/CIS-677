\section*{Problem 2}

First consider the case, where $S$ and $T$ only differ at one element, i.e. $|T| = |S| - 1$.

Consider the following algorithm: 

\begin{enumerate}
    \item Alice calculates the sum of all her elements $S_A$ and sends it to Bob
    \item Bob calculates the sum over all his elements $S_B$
    \item Bob outputs $S_A - S_B$
\end{enumerate}

$S_A$ is at most $\frac{n \cdot (n + 1)}{2} \leq n^2$ and can therefore be decoded with $2\cdot\log n$ bits. 

Since there is only one element $x \in S$ that is not in $T$, the difference of the sums must equal $x$. Therefore the algorithm outputs an element in $S\setminus T$.

Now, let $T$ and $S$ be arbitrary subsets of $[1..n]$ with $T \subset S$. 

Let $d = |S| - |T|$, let $n^*$ be the smallest power of 2 such that $n^* \geq n$.

Since Alice does not know $d$, she executes the following algorithm:

For $p \in \left\{1, \frac{1}{2}, \frac{1}{4}, \dots, \frac{1}{n^*}\right\}$:

\begin{enumerate}
    \item Bob and Alice choose $V \subset [1..n]$, $\forall i \in [1..n]: i \in V$ with probability $p$, from their shared source of randomness
    \item Alice computes $S_A^* = \operatorname{sum}(S \cap V)$
    \item Alice computes and sends $(S_A^*, |S \cap V|)$ 
    \item Bob: computes $S_B^* = \operatorname{sum}(S \cap T)$
    \item If $|S \cap V| = |T \cap V| + 1$: Bob returns $S_A^* - S_B^*$
\end{enumerate}

\textit{Note:} Instead of sending individual messages, all messages can be combined to one. Bob does his calculations after receiving this one message.

For one iteration the communication is in $\bigO(log n)$ bits, since both $S_A^*$ and $|A \cap S|$ require less than $2\cdot\log n$ bits. In total, there are $\log n^*$ repetitions, thus the total communication is in $\bigO(\log^2 n)$.

Since $(T \cap V) \subset (S \cap V)$, the results from the previous algorithm can be used. With this, a correct result is always returned if $|S \cap V| = |T \cap V| + 1$.

Thus we only need to show, that the probability of $|S \cap V| = |T \cap V| + 1$ is high enough for at least one iteration.

If $|S| - |T| = 1$, the algorithm succeeds with probability 1 in the first iteration. Otherwise the following analysis can be made:

There is one iteration where $p = \frac{1}{2^k}$ for each $k \in [2..\log n^*]$. Let $k'$ be such that $2^{k' -1} < d \leq 2^{k'}$. Let $c = 2^{k'}$. Then $\frac{c}{2} < d \leq c \Leftrightarrow \frac{c}{2} + 1 \leq d \leq c$ .

$
\Pr[\textit{Picking exactly 1 of the d elements when } p = 1/c] 
$

$ \geq
\Pr[\textit{Picking exactly 1 of } c/2 + 1 \textit{ elements when } p = 1/c] = \textit{ (by binomial distribution)}
$

$ = 
{c/2 + 1 \choose 1} \cdot \frac{1}{c} \cdot \left(1 - \frac{1}{c} \right)^{c/2} = 
(c/2 + 1)  \cdot \frac{1}{c} \cdot \sqrt{\left(1 - \frac{1}{c}\right)^c} \geq
\frac{1}{2} \cdot \sqrt{\left(1 - \frac{1}{c}\right)^c} \geq \textit{ (by using calculus and } c \geq 2)
$ 

$ \geq 
\frac{1}{2} \sqrt{\frac{1}{4}} = \frac{1}{4}
$

Therefore, the probability of failure is at most $1 - \frac{1}{4} = \frac{3}{4}$.

Instead of executing the algorithm once, we execute it $9 \cdot \log (1/\delta)$ times. The total communication is now in $\bigO(\log^2 n \log (1/\delta))$.

$\Pr[\textit{failure}] \leq 
\left(\frac{3}{4}\right)^{9 \cdot \log (1/\delta)} = 
e^{9 \cdot \log (1/\delta) \cdot \log(3/4)} \leq
e^{-\log(1/\delta)} = 
e^{\log \delta} =
\delta$

Thus, the success probability is at least $1 - \delta$.

\pagebreak

