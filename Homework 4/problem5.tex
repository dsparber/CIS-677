\section*{Problem 5}

\underline{High level outline:}

\begin{itemize}
    \item Use some computation, such that the expected value is equal to the total amount of annual sales.
    \item Use Chernoff bounds to analyse the probability, use the formulas for relative error with $\delta = 0.01 = 1\%$. 
    \item Do the computation $c \in \log (N + M))^{\bigO(1)}$ times with some random parameters and send the concatenation of all results to the headquarter. By repeating $c$ times the probability of getting the correct result is at least $99\%$.
\end{itemize}

If the result of each computation each server does is of order $(\log (N + M))^{\bigO(1)}$, then the total size of the single message each server sends is also bound by $(\log (N + M))^{\bigO(1)}$.

\underline{Computation:}

\textit{Idea 1:}

Each server picks some entries at random, computes the sum of sales and sends the result to the headquarter. The headquarter sums up all received values. 

We adjust the way elements are picked at random such that on expectation no value is missing or duplicated amongst all messages.

To have no duplicates we could use the shared source of randomness to assign stores to servers. If the server has the requested entry, it includes it into the sum.

The servers can also send the count of tuples that are part in their sum to the headquarter. The count can be sent with $\log M$ bits. The server can sum up the counts as well to check if all stores were included.

\underline{Analysis}

\textit{Idea 1:}

Let $X$ be a random variable. $X$ is consists of independent Bernoulli random variables has the following property:

$\EV[X] = \mu = $ total amount of annual sales.

$1 \cdot N \leq \mu \leq M \cdot N$


We now use Chernoff bounds ($\delta = 0.01$):

$\Pr[X > (1 + \delta) \EV[X]] \leq e^{-\frac{\mu\delta^2}{4}} \leq e^{-\frac{N}{40,000}}$

$\Pr[X < (1 - \delta) \EV[X]] \leq e^{-\frac{\mu\delta^2}{2}} \leq e^{-\frac{N}{20,000}}$

By repeating $c = 6 \cdot 40,000$ times, the probability of failure can be reduced to:

$\Pr[X > (1 + \delta)\mu] \leq e^{-\frac{c\cdot N}{40,000}} = e^{-6N} \leq e^{-6} <  0.005$

$\Pr[X < (1 - \delta)\mu] \leq e^{-\frac{c\cdot N}{20,000}} = e^{-12N} \leq e^{-6} <  0.005$


Therefore, the success probability is greater than $99\%$.