\subsection*{Part a - Deterministic algorithm}

Let $U = \{1,2,\dots,n\}$

For any integer $x$, $x_k$ is the $k$-th bit of $x$ in binary representations

\subsubsection*{Input}

$V \in \mathbb{R}_{_{\geq 0}}^n$, with at least one non-zero element.

\subsubsection*{Output}

Returns unique index $i$ such that $V[i] \neq 0$ whenever $V$ contains exactly one non-zero element. Otherwise outputs ``More than one non-zero entry''.

\subsubsection*{Algorithm}


\begin{enumerate}
    \item $a = 0$

    \item For $k \in \{0,\dots, \lceil \log n \rceil\}$:

    \begin{itemize}
        \item $S_k := \{s ~ | ~ s \in U \text{ if } s_k = 0 \}$, $\Bar{S_k} = U \setminus S_k$
                
        \item if $Q(S_k) \neq 0$ and $Q(\Bar{S_k}) \neq 0$: return ``More than one non-zero entry''
        
        \item $a_k = 1$ if $Q(S_k) \neq 0$ else $0$
        
    \end{itemize}
    
    \item return $a$
\end{enumerate}


\subsubsection*{Run time}

Run time in terms of summation queries:

Every iteration performs $\bigO(1)$ queries. There is a total number of $\lceil \log n \rceil$ iterations.

Thus, the algorithm performs $\bigO(\log n)$ summation queries.

\subsubsection*{Correctness}

\begin{itemize}
    \item Case 1: More than one non-zero entry
    
    There exists $i, j \in U$ with $i \neq j$, $V[i] > 0$ and $V[j] > 0$.
    
    Since $i \neq j$, there exists at least one index $k$ with $i_k \neq j_k$.
    
    Thus, only $i$ or $j$ is in $S_k$. Therefore, $Q(S_k) > 0$ and $Q(U \setminus S_k) > 0$
    
    The algorithm always outputs ``More than one non-zero entry''

    
    \item Case 2: Exactly on non-zero entry
    
    There exists a unique $b$ with $V[b] > 0$
    
    For every $k \in \{0,\dots, \lceil \log n \rceil\}$:
    
    \begin{itemize}
        \item Case $b_k = 0$: $b \notin S_k$, since $V[b]$ is the only non-zero element: $Q(S_k) = 0 \Rightarrow a_k = 0$
    
        \item Case $b_k = 1$: $b \in S_k \Rightarrow Q(S_k) = V[b] > 0 \Rightarrow a_k = 1$,
        
        \item Therefore $b_k = a_k$

    \end{itemize}
    
    $\Rightarrow a = b$
    
    Since $b$ is unique, the algorithm returns the only correct index.
    
\end{itemize}

\pagebreak