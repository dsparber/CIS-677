\section*{Problem 2}

\subsection*{Claim}

C = BPP


\subsection*{Proof}

By proofing C $\subseteq$ BPP and BPP $\subseteq$ C.

\subsubsection*{BPP $\subseteq$ C}

Let $L \in$ BPP be arbitrary.

By definition of BPP: There exists a poly-time algorithm A which on any input $x$ has the following behavior:

\begin{itemize}
    \item If $x \in L$, then $\operatorname{Pr}[\textit{A accepts } x] > \frac{3}{4}$
    
    \item If $x \notin L$, then $\operatorname{Pr}[\textit{A rejects } x] > \frac{3}{4}$
\end{itemize}

A is also a valid algorithm for C, since A fulfills the definition of C:

Let $\operatorname{poly}(n) = 4$:

\begin{itemize}
    \item If $x \in L$, then $\operatorname{Pr}[\textit{A accepts } x] \geq \frac{3}{4} = \frac{1}{2} + \frac{1}{4} = \frac{1}{2} + \frac{1}{\operatorname{poly}(n)}$
    
    \item If $x \notin L$, then $\operatorname{Pr}[\textit{A rejects } x] \geq \frac{3}{4} \geq \frac{1}{2}$
\end{itemize}

\subsubsection*{C $\subseteq$ BPP}

Let $L \in$ C be arbitrary.

By definition of C: There exists a poly-time algorithm A which on any input $x$ has the following behavior:

\begin{itemize}
    \item If $x \in L$, then $\operatorname{Pr}[\textit{A accepts } x] \geq \frac{1}{2} + \frac{1}{\operatorname{poly}(n)}$
    
    \item If $x \notin L$, then $\operatorname{Pr}[\textit{A rejects } x] \geq \frac{1}{2}$
\end{itemize}

Let B be an algorithm defined as follows:

\begin{itemize}
    \item Repeat algorithm A $k$ times. Accept, if at least $l = \frac{k}{2} + \frac{k}{2 \operatorname{poly}(n)}$ of executions of A accepted, otherwise reject.
\end{itemize}

Where $k = 32(\operatorname{poly}(n))^2\operatorname{log}(2)$

\underline{Claim}

B is a valid algorithm for BPP, i.e. B has the following properties:

\begin{itemize}
    \item If $x \in L$, then $\operatorname{Pr}[\textit{B accepts } x] > \frac{3}{4}$
    
    \item If $x \notin L$, then $\operatorname{Pr}[\textit{B rejects } x] > \frac{3}{4}$
\end{itemize}

\underline{Proof}

\begin{itemize}
    \item Assume $x \in L$:

Let $X_i$ be a 0/1 random variable, that indicates if the $i$-th execution of A accepted.

Let $X = \sum\limits_{i=1}^k X_i$,
$E[X] = k \cdot \left( \frac{1}{2} + \frac{1}{\operatorname{poly}(n)} \right) = \frac{k}{2} + \frac{k}{\operatorname{poly}(n)}$

$\operatorname{Pr}[\textit{B accepts } x] 
= \operatorname{Pr}[\textit{A accepts } x \textit{ at least l times}]
= \operatorname{Pr}[X \geq l] 
= 1 - \operatorname{Pr}[X < l]
= 1 - \operatorname{Pr}[X \leq l]$

$X$ consists of independent 0/1 random variables, thus Chernoff bounds with additive error can be applied:

$\operatorname{Pr}[X \leq l]= \operatorname{Pr}\left[X \leq  \frac{k}{2} + \frac{k}{2 \operatorname{poly}(n)}\right]
= \operatorname{Pr}\left[X \leq  \left(\frac{k}{2} + \frac{k}{\operatorname{poly}(n)}\right) - k\cdot\frac{1}{2\operatorname{poly}(n)}\right]
< e^{-\frac{\frac{1}{(2\operatorname{poly}(n))^2} k}{2}}
$

$=
e^{-\frac{\frac{1}{4 (\operatorname{poly}(n))^2} 32(\operatorname{poly}(n))^2\operatorname{log}(2)}{2}} 
=
e^{-\operatorname{log}(16)}
= \frac{1}{16} \leq \frac{1}{4}$

$\Rightarrow
\operatorname{Pr}[\textit{B accepts } x] > \frac{3}{4}$ 

\item Assume $x \notin L$:

Let $Y_i$ be a 0/1 random variable, that indicates if the $i$-th execution of A accepts.

Let $Y = \sum\limits_{i=1}^k Y_i$,
$E[Y] = k \cdot\frac{1}{2} = \frac{k}{2}$

$\operatorname{Pr}[\textit{B rejects } x] 
= \operatorname{Pr}[\textit{A accepts } x \textit{ less than l times}]
= \operatorname{Pr}[Y < l] 
= 1 - \operatorname{Pr}[Y \geq l]$

Applying Chernoff:

$\operatorname{Pr}[Y \geq l]
= \operatorname{Pr}\left[Y \geq \frac{k}{2} + \frac{k}{2 \operatorname{poly}(n)}\right]
= \operatorname{Pr}\left[Y \geq \frac{k}{2} + k\cdot\frac{1}{2\operatorname{poly}(n)}\right]
< e^{-\frac{\frac{1}{(2\operatorname{poly}(n))^2} k}{4}}$

$=
e^{-\frac{\frac{1}{4(\operatorname{poly}(n))^2} 32(\operatorname{poly}(n))^2\operatorname{log}(2)}{4}} 
=
e^{-\operatorname{log}(4)}
= \frac{1}{4}$

$\Rightarrow
\operatorname{Pr}[\textit{B rejects } x] > \frac{3}{4}$ 

\end{itemize}

Since A is a polynomial-time algorithm and A is executed $32(\operatorname{poly}(n))^2\operatorname{log}(2)$ times, B is a polynomial-time algorithm.

Thus, B is a valid algorithm for BPP.

\pagebreak

