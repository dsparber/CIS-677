\subsection*{Part b - Randomized algorithm}

\subsubsection*{Input}

$V \in \mathbb{R}_{_{\geq 0}}^n$, with exactly $d \in [1..n]$ non-zero elements, accuracy parameter $\delta$.

\subsubsection*{Output}

Returns an index $i$ with $V[i] \neq 0$ with probability of at least $1 - \delta$.

\subsubsection*{Algorithm}

\begin{itemize}
    \item Repeat $k$ times:
    \begin{enumerate}
        \item For $i \in [1..n]$: Draw $x_i \in \{0, 1\}$ with $\Pr[x_i = 1] = \frac{1}{d}$ 
        
        \item Let $S = \{i ~| ~ i \in [1..n] \text{ if } x_i = 1\}$
        
        \item If $Q(S) \neq 0$:
                
        \begin{itemize}
            \item Run deterministic algorithm with the following modification:
            
            Instead of using $S_k$, $S_k^\prime = \{s ~ | ~ s \in S_k \text{ if } x_s = 1\}$ is used.
            
            Instead of using $\Bar{S_k}$, $\Bar{S_k}^\prime = \{s ~ | ~ s \in \Bar{S_k} \text{ if } x_s = 1\}$ is used.
            
            \item If index is returned by deterministic algorithm: return index
            
        \end{itemize}
        
    \end{enumerate} 
    
    \item return $0$
\end{itemize} 

Where $k = 9 \cdot \log(1/\delta)$ and $Q_W(S) = \sum_{j \in S} W[j]$ is the summation query over vector $W$

\subsubsection*{Run time}

If $d = 1$, the algorithm runs in $\bigO(\log n)$.

Otherwise, one iteration of the algorithm performs one summation query plus at most $\bigO(\log n)$ summation queries by calling the deterministic algorithm.

Thus, one iteration performs $\bigO(\log n)$ summation queries.

By repeating $k \in \bigO(\log(1/\delta))$ times, the total number of summation queries is in  $\bigO(\log(n)\cdot\log(1/\delta))$.

\subsubsection*{Correctness}

\begin{itemize}
    \item Case $d = 1$:
    
    For every $i \in [1,\dots,n]$ $x_i = 1$ with probability $1$. Thus, the algorithm simply performs the deterministic algorithm. One iteration of the randomized algorithm is performed.

    The correctness follows from correctness of deterministic algorithm. The probability of returning the correct index is $1$ and therefore at least $1 - \delta$.
    
    \item Case $d \neq 1$
    
    Every index $i$, where $x_i = 0$, is never in any set $S_k^\prime$ and $\Bar{S_k}^\prime$. Thus, both summation queries never contains $V[i]$. 
    
    Likewise, every index $i$, where $x_i = 1$ is always in $S_k^\prime$ whenever $i$ is in $S_k$ and in $\Bar{S_k}^\prime$ whenever $i$ is in $\Bar{S_k}$
    
    Let $W = [V[0]\cdot x_0,\dots,V[n]\cdot x_n]$
    
    For this definition of $W$, performing the summation queries on $W$ with $S_k$ and $\Bar{S_k}$ is equivalent to performing the summation query on $V$ with $S_k^\prime$ and $\Bar{S_k}^\prime$. In the following steps, the equivalent problem, using $W$ is analyzed.
    
    
    \underline{Analyzing one iteration:}
    
    \textit{Claim:} $W$ contains exactly one non-zero element with probability $p \geq \frac{1}{4}$
    
    \textit{Proof:} 
    
    Every index with a zero element in $V$ is also an index with a zero element in $W$ by construction of $W$.
    
    Therefore, only non-zero elements of $V$ might remain in $W$. By construction, each of the $d$ non-zero element remains in $W$ with probability $\frac{1}{d}$. Let $X$ be the amount of non-zero elements remaining in $W$.
    
    $X \sim \operatorname{Binomial}\left(d, \frac{1}{d}\right)$
    
    $\Pr[X = 1] = 
    {d \choose 1} \cdot \frac{1}{d} \cdot \left(1 - \frac{1}{d}\right)^{d-1} = 
    \left(1 - \frac{1}{d}\right)^{d-1} =
    \frac{\left(1 - \frac{1}{d}\right)^{d} }{1 - \frac{1}{d}} \geq
    \frac{\left(1 - \frac{1}{d}\right)^{d} }{1 - 0} = 
    \left(1 - \frac{1}{d}\right)^{d} 
    $ 
    
    Since $d \in [1..n]$ and $d \neq 1$: $d \geq 2$ 

    For $d = 2$: $\left(1 - \frac{1}{d}\right)^{d} = \frac{1}{4}$
        
    The term $\left(1 - \frac{1}{d}\right)^{d}$ is monotonically increasing as $d$ increases (see calculus).  
    
    Thus, $\left(1 - \frac{1}{d}\right)^{d} \geq \frac{1}{4}$
    
    $\Rightarrow \Pr[X = 1] \geq \frac{1}{4}$
    
    \textit{q.e.d}
    
    The if condition ensures, that $W$ contains at least one non-zero element.
    
    Therefore, the input conditions for the deterministic algorithm are met. 
    
    Whenever $W$ contains exactly one non-zero element, the correct index of that element is returned by the deterministic algorithm. Otherwise, a string is returned.
    
    By construction of $W$, this index is also a valid non-zero index in the original vector. Hence, the entire algorithm succeeds and returns a valid index.
    
    The probability of $W$ containing exactly one non-zero element is $p \geq \frac{1}{4}$. Thus, one iteration succeeds with $p \geq \frac{1}{4}$.
    
    \underline{After $k$ iterations}
    
    The algorithm returns a correct element whenever one iteration returns.
    
    The algorithm can only fail, if no valid index was found within $k$ iterations, i.e. if every iteration fails.
    
    $\Rightarrow \Pr[\textit{error}] \leq 
    (1 - p)^k \leq 
    \left(1 - \frac{1}{4}\right)^k = 
    \left(\frac{3}{4}\right)^k = 
    e^{k \cdot \log(3/4)} = $
    
    $ = e^{9 \cdot \log(1/\delta) \cdot \log(3/4)} \leq
    e^{\frac{- 1}{\log(3/4)} \cdot \log(1/\delta) \cdot \log(3/4)} = 
    e^{-\log(1/\delta)} = 
    e^{\log \delta} =
    \delta$
    
    $\Rightarrow \Pr[\textit{success}] = 1 - \Pr[\textit{error}] \geq 1 - \delta$
    
\end{itemize}

