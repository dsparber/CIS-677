\section*{Problem 5}

\subsection*{Input}

Array $A[1..n]$ of $n$ distinct integers, parameter $\epsilon$.

\subsection*{Output}

If $A$ is sorted, returns YES.

If $A$ is $\epsilon$-far from being sorted, returns NO with probability at least $3/4$.

Otherwise return either YES or NO.

\subsection*{Algorithm}


\begin{enumerate}
    \item Repeat k times:

\begin{itemize}
    \item Pick $x \in A$ uniformly at random
    \item Search for $x$ with binary search.
    \item If search failed: return NO
\end{itemize}

\item Return YES
\end{enumerate}

Where $k = \frac{2}{\epsilon}$ 


\subsection*{Run time}

Binary search on an array with length $n$ has run time $\bigO(\log n)$. Thus, picking $x$ and checking whether the search failed can be done in $\bigO(1)$.
Thus one repetition runs in $\bigO(\log n)$.

Repeting $k = \frac{2}{\epsilon}$ times leads to a total run time of $\bigO\left(\frac{\log n}\epsilon{}\right)$.

\subsection*{Correctness}

\begin{itemize}
    \item Case 1: $A$ is sorted
    
    Finding $x \in A$ with binary search is always successful when $A$ is sorted. Thus the algorithm outputs YES. 
    
    \item Case 2: $A$ is $\epsilon$-far from being sorted
    
    \textit{To show:}  Algorithm returns NO with probability at least 3/4. 
    
    \underline{Analyzing one run of the loop body:}
        
    Lets $s_1, s_2,\dots,s_m$ be all the elements in $A$ that can be found by binary search.
    
    \textit{Claim:} $\forall i,j \in \{1,\dots,m\} : i < j \Rightarrow s_i < s_j$
    
    \textit{Proof:} Assume  $i < j$.
    
    By the way binary search works: There must exist an index $l$ where the binary search diverged. I.e. $\exists l : i < l$ and $l \leq j$.
    Since the search diverged at $l$: $s_i < s_l$ and $s_l \leq s_j$. By transitivity: $s_i < s_j$.
    
    Thus, $s_1, s_2,\dots,s_m$ is a sorted sequence. The longest sorted sequence has length $(1 - \epsilon) \cdot n$. Therefore: $m \leq (1 - \epsilon) \cdot n$.
    
    Drawing $x \in A$ uniformly at random: $x$ can only be found if $x = s_i$ for some $i \in \{1,\dots,m\}$.
    
    Therefore, the probability $p$ of drawing a $x$ which can be found is at most $ \frac{m}{n} = \frac{(1 - \epsilon) \cdot n}{n} = 1 - \epsilon$ 
    
     $\Rightarrow p \leq (1 - \epsilon)$
    
    \underline{Repeating the body of the loop $k$ times:}
    
    $\Pr[\textit{YES}] 
    = p^k 
    = (1 - \epsilon)^{k} 
    \leq e^{-\epsilon \cdot k}
    = e^{-\epsilon \cdot \frac{2}{\epsilon}}
    = e^{-2}
    = e^{\log\left(\frac{1}{4}\right)}
    = \frac{1}{4}$
    
    $\Pr[\textit{NO}] = 1 - \Pr[\textit{YES}] \geq 1 - \frac{1}{4} = \frac{3}{4}$
    
    Therefore, the algorithm yields NO with probability at least 3/4.
    
    \item Case 3: $A$ is neither sorted nor $\epsilon$ far from being sorted
    
    The algorithm terminates on every input. YES and NO are the only possible outputs. 
    Thus, the algorithm always outputs either YES or NO.
\end{itemize}